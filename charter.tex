\documentclass[
11pt, % The default document font size, options: 10pt, 11pt, 12pt
codirector, % Uncomment to add a codirector to the title page
]{charter} 




% El títulos de la memoria, se usa en la carátula y se puede usar el cualquier lugar del documento con el comando \ttitle
\titulo{Identificación de estados fenológicos de la flor de durazneros mediante visión por computadora} 

% Nombre del posgrado, se usa en la carátula y se puede usar el cualquier lugar del documento con el comando \degreename
%\posgrado{Carrera de Especialización en Sistemas Embebidos} 
%\posgrado{Carrera de Especialización en Internet de las Cosas} 
\posgrado{Carrera de Especialización en Inteligencia Artificial}
%\posgrado{Maestría en Sistemas Embebidos} 
%\posgrado{Maestría en Internet de las cosas}

% Tu nombre, se puede usar el cualquier lugar del documento con el comando \authorname
\autor{Ing. Héctor Luis Sánchez Márquez} 

% El nombre del director y co-director, se puede usar el cualquier lugar del documento con el comando \supname y \cosupname y \pertesupname y \pertecosupname
\director{Ing. Juan Ignacio Cavalieri}
\pertenenciaDirector{FIUBA} 
% FIXME:NO IMPLEMENTADO EL CODIRECTOR ni su pertenencia
\codirector{Esp. Lic. Nicolás Eduardo Horro} %para que aparezca en la portada se debe descomentar la opción codirector en el documentclass
\pertenenciaCoDirector{INVAP S.E.}

% Nombre del cliente, quien va a aprobar los resultados del proyecto, se puede usar con el comando \clientename y \empclientename
\cliente{Dr. Gerardo Sánchez}
\empresaCliente{Instituto Nacional de Tecnología Agropecuaria (INTA)}

% Nombre y pertenencia de los jurados, se pueden usar el cualquier lugar del documento con el comando \jurunoname, \jurdosname y \jurtresname y \perteunoname, \pertedosname y \pertetresname.
\juradoUno{Nombre y Apellido (1)}
\pertenenciaJurUno{pertenencia (1)} 
\juradoDos{Nombre y Apellido (2)}
\pertenenciaJurDos{pertenencia (2)}
\juradoTres{Nombre y Apellido (3)}
\pertenenciaJurTres{pertenencia (3)}
 
\fechaINICIO{22 de agosto de 2023}		%Fecha de inicio de la cursada de GdP \fechaInicioName
\fechaFINALPlan{10 de octubre de 2023} 	%Fecha de final de cursada de GdP
\fechaFINALTrabajo{abril de 2024}	%Fecha de defensa pública del trabajo final


\begin{document}

\maketitle
\thispagestyle{empty}
\pagebreak


\thispagestyle{empty}
{\setlength{\parskip}{0pt}
\tableofcontents{}
}
\pagebreak


\section*{Registros de cambios}
\label{sec:registro}


\begin{table}[ht]
\label{tab:registro}
\centering
\begin{tabularx}{\linewidth}{@{}|c|X|c|@{}}
\hline
\rowcolor[HTML]{C0C0C0} 
Revisión & \multicolumn{1}{c|}{\cellcolor[HTML]{C0C0C0}Detalles de los cambios realizados} & Fecha      \\ \hline
0      & Creación del documento.                                 &\fechaInicioName \\ \hline
1      & Se completa hasta el punto 5 inclusive.                 & 5 de septiembre de 2023 \\ \hline
2      & Aplicación de las correcciones de la revisión 1. \newline
Se completa hasta el punto 9 inclusive. & 12 de septiembre 2023 \\ \hline
3      & Aplicación de las correcciones de la revisión 2. \newline
Se completa hasta el punto 12 inclusive. & 18 de septiembre 2023 \\ \hline
4      & Aplicación de las correcciones de la revisión 3. \newline
Se completa hasta el punto 15 inclusive. & 25 de septiembre 2023 \\ \hline
4.1      & Aplicación de las correcciones de la revisión 4. & 1 de octubre 2023 \\ \hline
%		  Se puede agregar algo más \newline
%		  En distintas líneas \newline                                                    
%3      & Se completa hasta el punto 11 inclusive                & dd/mm/aaaa \\ \hline
%4      & Se completa el plan	                                 & dd/mm/aaaa \\ \hline
\end{tabularx}
\end{table}

\pagebreak



\section*{Acta de constitución del proyecto}
\label{sec:acta}

\begin{flushright}
Buenos Aires, \fechaInicioName
\end{flushright}

\vspace{2cm}

Por medio de la presente se acuerda con el \authorname\hspace{1px} que su Trabajo Final de la \degreename\hspace{1px} se titulará ``\ttitle'', consistirá esencialmente en desarrollar un algoritmo que identifique flores y su estado a partir de fotos de varetas, y tendrá un presupuesto preliminar estimado de 620 horas de trabajo y \$13.939.500 pesos argentinos de costos, con fecha de inicio \fechaInicioName\hspace{1px} y fecha de presentación pública \fechaFinalName.

Se adjunta a esta acta la planificación inicial.

\vfill

% Esta parte se construye sola con la información que hayan cargado en el preámbulo del documento y no debe modificarla
\begin{table}[ht]
\centering
\begin{tabular}{ccc}
\begin{tabular}[c]{@{}c@{}}Dr. Ing. Ariel Lutenberg \\ Director posgrado FIUBA\end{tabular} & \hspace{2cm} & \begin{tabular}[c]{@{}c@{}}\clientename \\ \empclientename \end{tabular} \vspace{2.5cm} \\ 
\multicolumn{3}{c}{\begin{tabular}[c]{@{}c@{}} \supname \\ Director del Trabajo Final\end{tabular}} \vspace{2.5cm} \\
%\begin{tabular}[c]{@{}c@{}}\jurunoname \\ Jurado del Trabajo Final\end{tabular}     &  & \begin{tabular}[c]{@{}c@{}}\jurdosname\\ Jurado del Trabajo Final\end{tabular}  \vspace{2.5cm}  \\
%\multicolumn{3}{c}{\begin{tabular}[c]{@{}c@{}} \jurtresname\\ Jurado del Trabajo Final\end{tabular}} \vspace{.5cm}                                                                     
\end{tabular}
\end{table}


\pagebreak

\section{1. Descripción técnica-conceptual del proyecto a realizar}
\label{sec:descripcion}

El Instituto Nacional de Tecnología Agropecuaria (INTA) es un organismo estatal descentralizado, con independencia operativa y financiera, que se encuentra adscrito a la Secretaría de Agricultura, Ganadería y Pesca del Ministerio de Economía de la Nación. Este ente nació en 1956 con el objetivo de impulsar la innovación y la transferencia de conocimientos en los sectores agroalimentario, agroindustrial y agropecuario a través de la investigación. Sus aportes permiten potenciar el país y generar nuevas oportunidades para acceder a mercados regionales e internacionales con productos y servicios de alto valor agregado.

\section{1.1 Introducción general}
\label{sec:descripcion}

La fenómica hace referencia a la obtención de un gran caudal de datos de las características de las plantas, lo que se denomina el fenotipo de la planta. Esta disciplina está en auge en la actualidad debido a sus aplicaciones potenciales. Por un lado, habilita el mejoramiento a gran escala debido a que es necesario vincular una gran cantidad de datos genéticos con datos fenotípicos para identificar la función de los genes. Por otro lado, si se incluyen otros conjuntos de datos como son los climáticos, permite realizar predicciones precisas sobre el comportamiento de las variedades, el cual es necesario para implementar lo que se conoce como agricultura de precisión. Sin embargo, la fruticultura no ha dado el salto hacia la fenómica.

En la  Estación Experimental Agropecuaria (EEA) de San Pedro se ha logrado secuenciar el ADN de más de 250 variedades de duraznero (Aballay et al., 2021, Scientific Reports) disponiendo de una base de datos genómica de 75 gigabases (Gb) de ADN. Esta base permite identificar genes que controlan características del duraznero mediante algoritmos de inteligencia artificial (IA). Además, se dispone de datos climáticos diarios que se toman de forma automática que incluyen: las temperaturas medias, precipitaciones, horas de frío, radiación, etc. Esta información se combina con los datos genómicos y posteriormente, con modelos de IA se predice el comportamiento de las variedades en escenarios climáticos futuros.

Por otro lado, las heladas primaverales son actualmente el mayor problema de los frutales a nivel mundial. Este fenómeno ocurre cuando las flores abiertas (estado “F”) se someten a temperaturas cercanas a los -2.5 °C. Por este motivo, es necesario conocer el número de flores que se encuentran en estado vulnerable ante un pronóstico de heladas, así como también la densidad de flores. Es del interés del INTA determinar el estado fenológico a campo y mejorar para la tolerancia a heladas. Para lograrlo, se empleará IA y visión por computadora.

La visión por computadora es un subdominio de la IA que permite a las máquinas imitar el sistema visual del ser humano. De esta forma, es posible extraer información a partir de imágenes. 

En la actualidad existen algoritmos capaces de detectar y clasificar de forma efectiva plantas a través de imágenes. Estos algoritmos se han utilizado para una gran variedad de aplicaciones, como es el caso de reconocimiento de enfermedades en plantas. Sin embargo, aún no se ha desarrollado un sistema para la extracción de estados fenológicos de la flor de durazneros.

La presente propuesta permitirá automatizar la toma de características de la flor de durazneros, a partir de fotos. De esta forma, se logrará aumentar el caudal de datos y mejorar los modelos de IA existentes.

\section{1.2 Marco de la propuesta}
\label{sec:descripcion}

La ejecución de este proyecto va alineada con el interés del INTA de determinar el estado de la flor de los durazneros a partir de imágenes.

Las especificaciones de las fotos de varetas que se disponen se detallan en la tabla 1. Por otro lado, las características a determinar se indican en la tabla 2. Se deberá evaluar diferentes modelos de IA a fin de definir el óptimo para esta tarea. Se deberá considerar la existencia de bases de datos de imágenes y/o modelos preentrenados disponibles.

\renewcommand{\tablename}{Tabla}
\begin{table}[ht]
\begin{center}
\begin{tabularx}{\textwidth}{| c | c | c | X | }
\hline
\rowcolor[HTML]{C0C0C0}
\multicolumn{4}{ |c| }{Características de las fotos de duraznos} \\ \hline
Formato & Número & Tamaño & Observaciones \\ \hline
JPG     & 250    & 1 MB   & Fotos de varetas de duraznero con flores en           diferentes estados fenológicos. La mayoría está en estado “F” pero también están en estado “E” y “G”. Cada foto tiene una regla. Cada foto tiene entre 10 a 12 varetas. \\ \hline
\end{tabularx}
\caption{Características de las fotos de duraznos.}
\label{tab:coches}
\end{center}
\end{table}

\renewcommand{\tablename}{Tabla}
\begin{table}[ht]
\begin{center}
\begin{tabularx}{\textwidth}{| c | X | }
\hline
\rowcolor[HTML]{C0C0C0}
\multicolumn{2}{ |c| }{Características de interés a ser determinadas} \\ \hline

  N° de Flores totales  & Presencia de Flor. \\ \hline
  Tipo de flor          & En las fotos existen dos tipos de flores: campanulácea y rosásea. \\ \hline
  Estado fenológico     & Discriminar entre estado “F” o flor abierta respecto a los demás estados (“E” flor cerrada o “G” flor sin pétalos). \\ \hline
  Densidad de flores    & Determinar la cantidad de flores por cm de vareta. \\ \hline
  
\end{tabularx}
\caption{Características de interés a ser determinadas.}
\end{center}
\end{table}

La solución que se propone hará uso de un modelo preentrenado para la detección de flores. Parte del modelado incluirá el ajuste de este algoritmo a las imágenes proporcionadas por el INTA. Como resultado final, se obtendrá un archivo en formato JSON con todos los datos que se detallan en la tabla 2.

La preparación del dataset será una etapa fundamental del proyecto donde se preprocesarán las imágenes, se etiquetarán y se les aplicará data augmentation. 

La figura 1 muestra esta solución a alto nivel.

\begin{figure}[htpb]
\centering 
\includegraphics[width=1\textwidth]{./Figuras/Tesis3.drawio.png}
\caption{Diagrama en bloques del sistema.}
\label{fig:diagBloques}
\end{figure}

\section{2. Identificación y análisis de los interesados}
\label{sec:interesados}

\begin{table}[ht]
%\caption{Identificación de los interesados}
%\label{tab:interesados}
\begin{tabularx}{\linewidth}{@{}|l|X|X|l|@{}}
\hline
\rowcolor[HTML]{C0C0C0} 
Rol           & Nombre y Apellido & Organización 	& Puesto 	\\ \hline
Auspiciante   & \clientename      &\empclientename  &    -    	\\ \hline
Cliente       & \clientename      &\empclientename	&    -   	\\ \hline
Responsable   & \authorname       & FIUBA        	& Alumno 	\\ \hline
Colaboradores & Dr. Maximiliano Aballay        & \empclientename &    -  	\\ \hline
Orientador    & \supname	      & \pertesupname 	& Director Trabajo final \\ \hline
Equipo        & Dr. Maximiliano Aballay \newline & \empclientename &    -     \\ \hline
Usuario final & INTA              &     INTA        &      -  	\\ \hline
\end{tabularx}
\end{table}



\section{3. Propósito del proyecto}
\label{sec:proposito}

El propósito de este proyecto es desarrollar un algoritmo que identifique flores de los durazneros y sus estados a partir de fotos de varetas, para aumentar el caudal de datos y mejorar los modelos de IA existentes.

\section{4. Alcance del proyecto}
\label{sec:alcance}

El proyecto incluye:
\begin{itemize}
	\item El preprocesamiento de las fotos para entrenar el modelo.
	\item La selección del modelo a entrenar.
	\item La elaboración del notebook de pruebas en Python.
	\item La implementación local del modelo.
\end{itemize}

El proyecto no incluye:
\begin{itemize}
	\item La recolección de datos/fotos.
	\item La integración con otros modelos que utilice el cliente.
\end{itemize}

\section{5. Supuestos del proyecto}
\label{sec:supuestos}
Para el desarrollo del presente proyecto se supone que:
\begin{itemize}
	\item Se contará con suficientes fotos de veretas para entrenar y evaluar el modelo. 
	\item Se contará con soporte para temas no referentes al área de inteligencia artificial que se necesiten para la elaboración de este proyecto.
	\item Se tendrán imágenes con la calidad adecuada para resolver el problema planteado.
	\item Se contará con los recursos de hardware necesarios para el entrenamiento del modelo.
	\item El código se desarrollará en una notebook de Google Colab.
\end{itemize}

\section{6. Requerimientos}
\label{sec:requerimientos}

\begin{enumerate}
	\item Requerimientos funcionales
		\begin{enumerate}
			\item El sistema tomará como entrada imágenes de varetas de durazneros en formato JPG.			
			\item El algoritmo debe detectar la presencia de las varetas de los durazneros e identificar el tipo de flor que posee.			        
			\item El algoritmo debe identificar el estado fenológico de cada flor de duraznero en la vareta. Este estado puede ser F de flor abierta, E de flor cerrada o G de flor sin pétalos. 
			\item El algoritmo debe determinar la cantidad de flores por centímetro de vareta.
			\item El sistema debe entregar como resultado un archivo en formato JSON con los datos detectados por el algoritmo y una imagen donde se puedan visualizar las detecciones.
			\item El sistema debe funcionar en una computadora local.
		\end{enumerate}
	\item Requerimientos de diseño e implementación
		\begin{enumerate}
			\item El diseño debe ser modular.
			\item El algoritmo se elaborará en una notebook de Google Colab, utilizando el lenguaje de  programación Python y bibliotecas de IA correspondientes. 
		\end{enumerate}
	\item Requerimiento de evaluación y prueba
	\begin{enumerate}
			\item El modelo se evaluará con imágenes provenientes del mismo dataset de imágenes entregado por el cliente.
			 \item Las métricas que se utilizarán para la evaluación del modelo serán: \textit{mean average precision} (mAP), matriz de confusión, \textit{precision}, \textit{recall}, \textit{F1}, \textit{accuracy}.
		\end{enumerate}
	\item Requerimientos de documentación
	\begin{enumerate}
			\item El funcionamiento del sistema debe estar correctamente explicado y documentado.
			 \item El código en la notebook estará correctamente comentado como parte de buenas prácticas del desarrollo de software.
			 \item Inclusión de documentación en un repositorio, mediante un archivo README.md (opcional).
		\end{enumerate}
\end{enumerate}

\section{7. Historias de usuarios (\textit{Product backlog})}
\label{sec:backlog}

Se identifican los siguientes roles:

\begin{itemize}
	\item Científico: tiene dominio en el problema a resolver. Es el que hará uso de los resultados proporcionados por el algoritmo. Es el que utiliza y proporciona las fotos de las varetas de durazneros. 
	\item Desarrollador de IA: es el que se encarga de mantener el modelo y de desarrollar nuevas funcionalidades.
\end{itemize}

Para estimar el puntaje de cada historia de usuario se consideran tres aspectos:

\begin{itemize}
	\item Esfuerzo:
	\begin{itemize}
	\item Bajo (1). 
	\item Medio (3). 
	\item Alto (5).
\end{itemize}
	\item Complejidad: 
	\begin{itemize}
	\item Bajo (1).
	\item Medio (3). 
	\item Alto (5).
\end{itemize}
	\item Riesgo:
	\begin{itemize}
	\item Bajo (1). 
	\item Medio (3). 
	\item Alto (5).
\end{itemize}
\end{itemize}

Donde el puntaje final es el número de Fibonacci más próximo a la suma de los puntajes parciales.

Se definen las siguientes historias de usuario:

\begin{itemize}
	\item Como científico, quiero extraer las características de flores de durazneros a través de imágenes usando un modelo de IA, para automatizar este proceso que actualmente se lo realiza de forma manual.
	\begin{itemize}
	\item Esfuerzo (3): porque el tiempo que se requiere para el desarrollo del modelo y su correcto funcionamiento puede ser medianamente largo.
	\item Complejidad (5): porque se requiere de un sólido conocimiento en modelos de aprendizaje basados en visión por computadora. 
	\item Riesgo (5): porque se pueden presentar imprevistos con las imágenes proporcionadas y compatibilidad con el modelo.
	\item Story Point: 13.
\end{itemize}
	\item Como desarrollador de IA, quiero contar con un código ordenado que siga las buenas prácticas del desarrollo del software, que esté bien documentado y que esté correctamente versionado, para poder continuar con su mantenimiento y mejora a través del tiempo.
	\begin{itemize}
	\item Esfuerzo (3): porque el tiempo que se emplea en acomodar el código y dejar todo bien documentado es moderado.
	\item Complejidad (3): porque puede incluir una refactorización del código original hecho en la notebook.
	\item Riesgo (1): porque no se esperan imprevistos en esta etapa.
	\item Story Point: 8.
\end{itemize}
	\item Como científico, quiero que el modelo de aprendizaje se pueda ejecutar de forma local en mi computadora de escritorio o laptop para no hacer uso de servidores externos que puedan incrementar el costo del proyecto.
	\begin{itemize}
	\item Esfuerzo (3): porque el tiempo que se emplea en crear un \textit{endpoint} local es moderado.
	\item Complejidad (3): porque existen bibliotecas que facilitan la creación de \textit{endpoints} simples de forma muy rápida e intuitiva.
	\item Riesgo (1): porque no se esperan imprevistos en esta etapa.
	\item Story Point: 8.
\end{itemize}
\end{itemize}

\section{8. Entregables principales del proyecto}
\label{sec:entregables}

Los entregables del proyecto son:

\begin{itemize}
	\item Esquema del funcionamiento de cada pieza del \textit{pipeline} de IA.
	\item Notebook con el código del algoritmo.
	\item Informe de avance.
	\item Informe final/memoria técnica del proyecto.
\end{itemize}

\section{9. Desglose del trabajo en tareas}
\label{sec:wbs}

\begin{enumerate}
\item Preparación de los datos. (160 h)
	\begin{enumerate}
	\item Análisis preliminar de las imágenes. (30 h)
	\item Estudio y búsqueda de herramientas de etiquetado de elementos en imágenes. (35 h)
	\item Etiquetado de imágenes. (40 h)
	\item Aplicar \textit{data augmentation}. (25 h)
	\item Aplicar redimensionamiento. (20 h)
	\item Cargar dataset. (10 h)
	\end{enumerate}
\item Desarrollo del modelo de IA. (200 h)
	\begin{enumerate}
	\item Estudio y búsqueda de modelos preentrenados. (40 h)
	\item Estudio de arquitecturas de \textit{deep learning} viables. (40 h)
	\item Desarrollo del modelo usando \textit{transfer learning}. (40 h)
	\item Entrenamiento del modelo. (40 h)
	\item Desarrollo de \textit{endpoint}. (40 h)
	\end{enumerate}
\item Prueba de desempeño del modelo. (120 h)
	\begin{enumerate}
	\item Pruebas sobre datos de validación.  (40 h)
	\item Pruebas sobre el \textit{endpoint} del modelo. (40 h)
	\item Ajuste y optimización de hiperparámetros. (40 h)
	\end{enumerate}
\item Tareas de documentación. (140 h)
	\begin{enumerate}
	\item Elaboración de diagramas. (20 h)
	\item Redacción del informe de avance. (40 h)
	\item Redacción de la memoria final del proyecto. (40 h)
	\item Preparación de presentación del proyecto. (40 h)
	\end{enumerate}
\end{enumerate}

Cantidad total de horas: 620 h.

\section{10. Diagrama de Activity On Node}
\label{sec:AoN}


%La figura \ref{fig:AoN} fue elaborada con el paquete latex tikz y pueden consultar la siguiente referencia \textit{online}:

%\url{https://www.overleaf.com/learn/latex/LaTeX_Graphics_using_TikZ:_A_Tutorial_for_Beginners_(Part_3)\%E2\%80\%94Creating_Flowcharts}

La figura 2 ilustra el diagrama de \textit{Activity on Node}. Las tareas están agrupadas por colores y siguiendo la estructura detallada en la sección anterior:

\begin{enumerate}
	\item Celeste: preparación de los datos.
	\item Amarillo: desarrollo del modelo de IA.
	\item Lila: prueba de desempeño del modelo.
	\item Verde: tareas de documentación.
\end{enumerate}

El camino crítico, cuya duración es de 450 horas, se muestra resaltado en color rojo.


\begin{figure}[htpb]
\centering 
\includegraphics[width=.8\textwidth]{./Figuras/AoN3.png}
\caption{Diagrama de \textit{Activity on Node}.}
\label{fig:AoN}
\end{figure}


\pagebreak

\section{11. Diagrama de Gantt}
\label{sec:gantt}

En el diagrama de \textit{Activity on Node}, se muestra que algunas tareas pueden ser paralelizables. Sin embargo, en el diagrama de Gantt se mostrarán de forma secuencial debido a que todas estas tareas las estará llevando a cabo el responsable del proyecto. Esto quiere decir, que solo una persona las estará realizando.

El diagrama de Gantt se muestra separado en dos partes. La figura 3 contiene el desglose de tareas y la figura 4 contiene el diagrama propiamente dicho.

\begin{figure}[htpb]
\centering 
\includegraphics[width=.8\textwidth]{./Figuras/TableG3.png}
\caption{Desglose de tareas.}
\label{fig:diagBloques}
\end{figure}


\begin{landscape}
\begin{figure}[htpb]
\centering 
\includegraphics[height=.85\textheight]{./Figuras/Gantt.png}
\caption{Diagrama de Gantt.}
\label{fig:diagGantt}
\end{figure}

\end{landscape}



\section{12. Presupuesto detallado del proyecto}
\label{sec:presupuesto}

A continuación se presenta el presupuesto detallado del proyecto expresado en pesos argentinos:

\begin{table}[htpb]
\centering
\begin{tabularx}{\linewidth}{@{}|X|c|r|r|@{}}
\hline
\rowcolor[HTML]{C0C0C0} 
\multicolumn{4}{|c|}{\cellcolor[HTML]{C0C0C0}COSTOS DIRECTOS} \\ \hline
\rowcolor[HTML]{C0C0C0} 
Descripción &
  \multicolumn{1}{c|}{\cellcolor[HTML]{C0C0C0}Cantidad} &
  \multicolumn{1}{c|}{\cellcolor[HTML]{C0C0C0}Valor unitario} &
  \multicolumn{1}{c|}{\cellcolor[HTML]{C0C0C0}Valor total} \\ \hline
   \multicolumn{1}{|l|}{Horas de ingeniería} 
 &
  \multicolumn{1}{c|}{620} &
  \multicolumn{1}{c|}{\$21.875} &
  \multicolumn{1}{c|}{\$13.562.500} \\ \hline
   \multicolumn{1}{|l|}{Computadora personal} 
 &
  \multicolumn{1}{c|}{1} &
  \multicolumn{1}{c|}{\$300.000} &
  \multicolumn{1}{c|}{\$300.000} \\ \hline
\multicolumn{3}{|c|}{SUBTOTAL} &
  \multicolumn{1}{c|}{\$13.862.500} \\ \hline
\rowcolor[HTML]{C0C0C0} 
\multicolumn{4}{|c|}{\cellcolor[HTML]{C0C0C0}COSTOS INDIRECTOS} \\ \hline
\rowcolor[HTML]{C0C0C0} 
Descripción &
  \multicolumn{1}{c|}{\cellcolor[HTML]{C0C0C0}Cantidad} &
  \multicolumn{1}{c|}{\cellcolor[HTML]{C0C0C0}Valor unitario} &
  \multicolumn{1}{c|}{\cellcolor[HTML]{C0C0C0}Valor total} \\ \hline
\multicolumn{1}{|l|}{Licencia de Roboflow} &
   \multicolumn{1}{|c|}{1}
   &
   \multicolumn{1}{|c|}{\$70.000}
   &
   \multicolumn{1}{|c|}{\$70.000}
   \\ \hline
\multicolumn{1}{|l|}{Google Colab pro} &
\multicolumn{1}{|c|}{1}
   &
   \multicolumn{1}{|c|}{\$7.000}
   &
   \multicolumn{1}{|c|}{\$7.000}
   \\ \hline
\multicolumn{3}{|c|}{SUBTOTAL} &
  \multicolumn{1}{c|}{\$77.000} \\ \hline
\rowcolor[HTML]{C0C0C0}
\multicolumn{3}{|c|}{TOTAL} &  \multicolumn{1}{c|}{\$13.939.500} 
   \\ \hline
\end{tabularx}%
\end{table}


\section{13. Gestión de riesgos}
\label{sec:riesgos}

a) Identificación de los riesgos y estimación de sus consecuencias: en esta sección se listan los riesgos que podrían afectar negativamente los planes previstos para este proyecto.

Para el análisis se asignan las siguientes valoraciones:

\begin{itemize}
\item Severidad (S) de 1 (bajo) a 10 (alto).
\item Probabilidad de ocurrencia (O) de 1 (bajo) a 10 (alto).
\end{itemize}

Riesgo 1. Datos insuficientes. La cantidad de imágenes proporcionadas es insuficiente o su calidad no es la adecuada para entrenar el modelo.

\begin{itemize}
\item Severidad (S): 10. El éxito del modelo esta directamente relacionado con la calidad y cantidad de imágenes que se le proporcione. 

\item Probabilidad de ocurrencia (O): 8. El INTA cuenta con un conjunto pequeño de imágenes. 
\end{itemize}

Riesgo 2. Etiquetar incorrectamente los estados fenológicos de la flor por falta de conocimiento en el área.

\begin{itemize}
\item Severidad (S): 10. El mal etiquetado del conjunto de imágenes influirá en la capacidad del modelo de poder identificar correctamente el estado de las flores. 

\item Probabilidad de ocurrencia (O): 4. El INTA estará colaborando activamente en el etiquetado para aportar su conocimiento experto en el área. 
\end{itemize}

Riesgo 3. Falta de recursos para entrenar el modelo de aprendizaje.

\begin{itemize}
\item Severidad (S): 10. La disponibilidad de recursos de computo y memoria es esencial para que el modelo pueda finalizar su entrenamiento y pueda ejecutar sus predicciones. 

\item Probabilidad de ocurrencia (O): 7. Al usar Google Colab es probable encontrar falta de disponibilidad de recursos. 
\end{itemize}

Riesgo 4. Bajo desempeño del modelo de aprendizaje para detectar los estados de la flor.

\begin{itemize}
\item Severidad (S): 8. No poder detectar correctamente los estados de la flor de duraznero conlleva al no cumplimiento del objectivo principal. 

\item Probabilidad de ocurrencia (O): 3. La probabilidad de que esto ocurra es baja porque actualmente existen muchos trabajos de investigación que han logrado la detección de plantas en imágenes, lo cual aumenta la posibilidad de que este algoritmo funcione correctamente. 
\end{itemize}

Riesgo 5. Mala interpretación de los requerimientos del cliente y que el algoritmo no preste el servicio esperado.

\begin{itemize}
\item Severidad (S): 5. Si el algoritmo no funciona como se desea debido a una mala interpretación, es posible reentrenar el modelo o incluso reemplazarlo, pero afectaría directamente al cronograma del proyecto. 

\item Probabilidad de ocurrencia (O): 3. La probabilidad de que esto ocurra es baja porque la comunicación con el cliente será continua y los avances se irán informando cada 2 semanas, de forma de detectar cualquier error de este estilo. 
\end{itemize}

b) Tabla de gestión de riesgos:      (El RPN se calcula como RPN= S x O)

\begin{table}[htpb]
\centering
\begin{tabularx}{\linewidth}{@{}|X|c|c|c|c|c|c|@{}}
\hline
\rowcolor[HTML]{C0C0C0} 
Riesgo & S & O & RPN & S* & O* & RPN* \\ \hline
1. Datos insuficientes.      &  10 & 8 &  \textcolor{red}{80}  &  6  &  5  &   30   \\ \hline
2. Etiquetar incorrectamente los estados fenológicos de la flor.      &  10 & 4  &     \textcolor{red}{40} & 10   & 2   &  20    \\ \hline
3. Falta de recursos.       &  10 & 7  &  \textcolor{red}{70}   &  10  & 2   & 20      \\ \hline
4. Bajo desempeño del modelo de aprendizaje.       &  8 & 3  & 24  &  -  &  -  &     - \\ \hline
5. Mala interpretación de los requerimientos.       & 5  & 3  &  15   &  -  &   - &     - \\ \hline
\end{tabularx}%
\end{table}

Criterio adoptado: 
se tomarán medidas de mitigación en los riesgos cuyos números de RPN sean mayores a 30.

Nota: los valores marcados con (*) en la tabla corresponden luego de haber aplicado la mitigación.

c) Plan de mitigación de los riesgos que originalmente excedían el RPN máximo establecido:

Riesgo 1. 

Plan de mitigación: para mitigar este riesgo se tienen tres alternativas. Por un lado, se puede utilizar algoritmos preentrenados usando \textit{transfer learning}, se puede utilizar \textit{data augmentation} o se pueden tomar más imágenes de varetas de durazanero. Ninguna de estas opciones son excluyentes entre sí.

\begin{itemize}
\item Severidad (S*): 6. La severidad disminuye debido a que el uso de modelos preentrenados en este tipo de problemas normalmente logra minimizar los efectos de la insuficiencia de datos.

\item Probabilidad de ocurrencia (O*): 5. Las probabilidades de que ocurra este riesgo disminuye debido a que esta sugerencia es un procedimiento probado y funcional. 
\end{itemize}

Riesgo 2. 

Plan de mitigación: para mitigar este riesgo, se tendrá ayuda de los especialistas del lado del cliente para confirmar que el etiquetado de los estados de la flor en las imágenes sea correcto.

\begin{itemize}
\item Severidad (S): 10. La severidad se mantiene.

\item Probabilidad de ocurrencia (O*): 2. La probabilidad de que ocurra este riesgo disminuye porque se tendrá una confirmación o verificación del lado del cliente. 
\end{itemize}

Riesgo 3. 

Plan de mitigación: para mitigar este riesgo, se utilizará Google Colab pro que cuenta con una mayor disponibilidad de recursos, incluyendo el uso de GPU.

\begin{itemize}
\item Severidad (S): 10. La severidad se mantiene.

\item Probabilidad de ocurrencia (O*): 2. La probabilidad de que ocurra este riesgo disminuye al aumentar la cantidad de recursos disponibles con la versión pro de Google Colab. 
\end{itemize}

\section{14. Gestión de la calidad}
\label{sec:calidad}

Req \#1.2. El algoritmo debe detectar la presencia de las varetas de los durazneros e identificar el tipo de flor que posee.

\begin{itemize}
	\item Verificación: ejecutar el modelo y verificar visualmente  que a la salida, tanto la imagen como el archivo en formato JSON contienen la correcta detección de la vareta y su tipo de flor.
	
	\item Validación: procesar una foto de vareta de duraznero con el modelo de IA y obtener a la salida el tipo de flor que posee la vareta. 
\end{itemize}

Req \#1.3. El algoritmo debe identificar el estado fenológico de cada flor de duraznero en la vareta. Este estado puede ser F de flor abierta, E de flor cerrada o G de flor sin pétalos.

\begin{itemize}
	\item Verificación: ejecutar el modelo y verificar visualmente  que a la salida, tanto la imagen como el archivo en formato JSON contienen los estados de las flores correctamente identificados.
	
	\item Validación: procesar una foto de vareta de duraznero con el modelo de IA y obtener a la salida una imagen y un archivo en formato JSON con la información de los estados fenológicos de las flores. 
\end{itemize}

Req \#1.4. El algoritmo debe determinar la cantidad de flores por centímetro de vareta.

\begin{itemize}
	\item Verificación: ejecutar el modelo y comprobar en el archivo en formato JSON de salida si el número total de flores detectadas por centímetro de vareta coincide con lo que se visualiza en la imagen.
	
	\item Validación: procesar una foto de vareta de duraznero con el modelo de IA y extraer la cantidad de flores por centímetro de vareta sin errores. 
\end{itemize}

Req \#1.5. El sistema debe entregar como resultado un archivo en formato JSON con los datos detectados por el algoritmo y una imagen donde se puedan visualizar las detecciones.

\begin{itemize}
	\item Verificación: ejecutar el modelo y verificar que a la salida se obtiene un archivo en formato JSON y una imagen con las detecciones.
	
	\item Validación: procesar una foto de vareta de duraznero con el algoritmo y obtener un archivo en formato JSON con las características de las flores y adicionalmente la imagen con las detecciones. 
\end{itemize}

Req \#1.6. El sistema debe funcionar en una computadora local.

\begin{itemize}
	\item Verificación: dejar funcionando el modelo de aprendizaje en una computadora local y que este no arroje ningún tipo de error.
	
	\item Validación: activar el modelo de aprendizaje en la computadora local y empezar a correr inferencias. 
\end{itemize}

Req \#2.1. El diseño debe ser modular.

\begin{itemize}
	\item Verificación: visualizar en la \textit{notebook} que cada parte del \textit{pipeline} de procesamiento esté correctamente seccionado por funciones.
	
	\item Validación: inspeccionar que el código sea modular. 
\end{itemize}

Req \#3.2. Las métricas que se utilizarán para la evaluación del modelo serán: \textit{mean average precision} (mAP), matriz de confusión, \textit{precision}, \textit{recall}, \textit{F1}, \textit{accuracy}.

\begin{itemize}
	\item Verificación: visualizar que al final del entrenamiento del modelo, las métricas con las que se evalua al sistema, sean las mencionadas en los requerimientos.
	
	\item Validación: inspeccionar en la \textit{notebook} las métricas que definen el rendimiento del mejor modelo. 
\end{itemize}

Req \#4.1. El funcionamiento del sistema debe estar correctamente explicado y documentado.

\begin{itemize}
	\item Verificación: no aplica.
	
	\item Validación: inspeccionar la documentación proporcionada. 
\end{itemize}

Req \#4.2. El código en la \textit{notebook} estará correctamente comentado como parte de buenas prácticas del desarrollo de software.

\begin{itemize}
	\item Verificación: revisar que cada sección del código este correctamente comentada.
	
	\item Validación: inspeccionar los comentarios en el código y verificar que sean comprensibles. 
\end{itemize}

\section{15. Procesos de cierre}    
\label{sec:cierre}

A continuación se describen las pautas que darán cierre al proyecto:

\begin{itemize}
	\item Pautas de trabajo que se seguirán para analizar si se respetó el Plan de Proyecto original:
	
	 \begin{itemize}
	 \item El responsable del proyecto comparará la planificación inicial del cronograma y los tiempos de ejecución reales del proyecto.
	 
	 \item El responsable del proyecto verificará si las tareas planteadas inicialmente correspondieron a la ejecución real.
	  \end{itemize}
	\item Identificación de las técnicas y procedimientos útiles e inútiles que se emplearon, y los problemas que surgieron y cómo se solucionaron:
	
	 \begin{itemize}
	 \item El responsable del proyecto con ayuda de su director examinarán los procesos fallidos y exitosos que se dieron durante el desarrollo del proyecto.
	 
	  \item El responsable del proyecto documentará todos los procesos fallidos y exitosos que se obtuvieron. 
	  
	  \end{itemize}
	\item Indicar quién organizará el acto de agradecimiento a todos los interesados, y en especial al equipo de trabajo y colaboradores:
	
	  \begin{itemize}
	 \item Posterior a la defensa pública, el responsable del proyecto se encargará de dar el reconocimiento al mérito de cada uno de los involucrados en este proyecto.
	 
	  \end{itemize}
\end{itemize}


\end{document}
