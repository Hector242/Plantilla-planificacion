\documentclass[
11pt, % The default document font size, options: 10pt, 11pt, 12pt
codirector, % Uncomment to add a codirector to the title page
]{charter} 




% El títulos de la memoria, se usa en la carátula y se puede usar el cualquier lugar del documento con el comando \ttitle
\titulo{Identificación de estados fenelógicos de la flor de durazneros mediante visión por computadora} 

% Nombre del posgrado, se usa en la carátula y se puede usar el cualquier lugar del documento con el comando \degreename
%\posgrado{Carrera de Especialización en Sistemas Embebidos} 
%\posgrado{Carrera de Especialización en Internet de las Cosas} 
\posgrado{Carrera de Especialización en Inteligencia Artificial}
%\posgrado{Maestría en Sistemas Embebidos} 
%\posgrado{Maestría en Internet de las cosas}

% Tu nombre, se puede usar el cualquier lugar del documento con el comando \authorname
\autor{Ing. Héctor Luis Sánchez Márquez} 

% El nombre del director y co-director, se puede usar el cualquier lugar del documento con el comando \supname y \cosupname y \pertesupname y \pertecosupname
\director{Nombre del Director}
\pertenenciaDirector{pertenencia} 
% FIXME:NO IMPLEMENTADO EL CODIRECTOR ni su pertenencia
\codirector{Maximiliano Aballay} % para que aparezca en la portada se debe descomentar la opción codirector en el documentclass
\pertenenciaCoDirector{INTA}

% Nombre del cliente, quien va a aprobar los resultados del proyecto, se puede usar con el comando \clientename y \empclientename
\cliente{Dr. Gerardo Sánchez}
\empresaCliente{Instituto Nacional de Tecnología Agropecuaria (INTA)}

% Nombre y pertenencia de los jurados, se pueden usar el cualquier lugar del documento con el comando \jurunoname, \jurdosname y \jurtresname y \perteunoname, \pertedosname y \pertetresname.
\juradoUno{Nombre y Apellido (1)}
\pertenenciaJurUno{pertenencia (1)} 
\juradoDos{Nombre y Apellido (2)}
\pertenenciaJurDos{pertenencia (2)}
\juradoTres{Nombre y Apellido (3)}
\pertenenciaJurTres{pertenencia (3)}
 
\fechaINICIO{22 de agosto de 2023}		%Fecha de inicio de la cursada de GdP \fechaInicioName
\fechaFINALPlan{10 de octubre de 2023} 	%Fecha de final de cursada de GdP
\fechaFINALTrabajo{TBD}	%Fecha de defensa pública del trabajo final


\begin{document}

\maketitle
\thispagestyle{empty}
\pagebreak


\thispagestyle{empty}
{\setlength{\parskip}{0pt}
\tableofcontents{}
}
\pagebreak


\section*{Registros de cambios}
\label{sec:registro}


\begin{table}[ht]
\label{tab:registro}
\centering
\begin{tabularx}{\linewidth}{@{}|c|X|c|@{}}
\hline
\rowcolor[HTML]{C0C0C0} 
Revisión & \multicolumn{1}{c|}{\cellcolor[HTML]{C0C0C0}Detalles de los cambios realizados} & Fecha      \\ \hline
0      & Creación del documento                                 &\fechaInicioName \\ \hline
1      & Se completa hasta el punto 5 inclusive                 & 5 de septiembre de 2023 \\ \hline
2      & Aplicación de las correcciones de la revisión 1. \newline
Se completa hasta el punto 9 inclusive & 12 de septiembre 2023 \\ \hline
%		  Se puede agregar algo más \newline
%		  En distintas líneas \newline                                                    
%3      & Se completa hasta el punto 11 inclusive                & dd/mm/aaaa \\ \hline
%4      & Se completa el plan	                                 & dd/mm/aaaa \\ \hline
\end{tabularx}
\end{table}

\pagebreak



\section*{Acta de constitución del proyecto}
\label{sec:acta}

\begin{flushright}
Buenos Aires, \fechaInicioName
\end{flushright}

\vspace{2cm}

Por medio de la presente se acuerda con el \authorname\hspace{1px} que su Trabajo Final de la \degreename\hspace{1px} se titulará ``\ttitle'', consistirá esencialmente en desarrollar un algoritmo que identifique flores y su estado a partir de fotos de varetas, y tendrá un presupuesto preliminar estimado de 600 horas de trabajo y \textcolor{red}{\$XXX}, con fecha de inicio \fechaInicioName\hspace{1px} y fecha de presentación pública \fechaFinalName.

Se adjunta a esta acta la planificación inicial.

\vfill

% Esta parte se construye sola con la información que hayan cargado en el preámbulo del documento y no debe modificarla
\begin{table}[ht]
\centering
\begin{tabular}{ccc}
\begin{tabular}[c]{@{}c@{}}Dr. Ing. Ariel Lutenberg \\ Director posgrado FIUBA\end{tabular} & \hspace{2cm} & \begin{tabular}[c]{@{}c@{}}\clientename \\ \empclientename \end{tabular} \vspace{2.5cm} \\ 
\multicolumn{3}{c}{\begin{tabular}[c]{@{}c@{}} \supname \\ Director del Trabajo Final\end{tabular}} \vspace{2.5cm} \\
%\begin{tabular}[c]{@{}c@{}}\jurunoname \\ Jurado del Trabajo Final\end{tabular}     &  & \begin{tabular}[c]{@{}c@{}}\jurdosname\\ Jurado del Trabajo Final\end{tabular}  \vspace{2.5cm}  \\
%\multicolumn{3}{c}{\begin{tabular}[c]{@{}c@{}} \jurtresname\\ Jurado del Trabajo Final\end{tabular}} \vspace{.5cm}                                                                     
\end{tabular}
\end{table}


\pagebreak

\section{1. Descripción técnica-conceptual del proyecto a realizar}
\label{sec:descripcion}

El Instituto Nacional de Tecnología Agropecuaria (INTA) es un organismo estatal descentralizado, con independencia operativa y financiera, que se encuentra adscrito a la Secretaría de Agricultura, Ganadería y Pesca del Ministerio de Economía de la Nación. Este ente nació en 1956 con el objetivo de impulsar la innovación y la transferencia de conocimientos en los sectores agroalimentario, agroindustrial y agropecuario a través de la investigación. Sus aportes permiten potenciar el país y generar nuevas oportunidades para acceder a mercados regionales e internacionales con productos y servicios de alto valor agregado.

\section{1.1 Introducción general}
\label{sec:descripcion}

La fenómica hace referencia a la obtención de un gran caudal de datos de las características de las plantas, lo que se denomina el fenotipo de la planta. Esta disciplina está en auge en la actualidad debido a sus aplicaciones potenciales. Por un lado, habilita el mejoramiento a gran escala debido a que es necesario vincular una gran cantidad de datos genéticos con datos fenotípicos para identificar la función de los genes. Por otro lado, si se incluyen otros conjuntos de datos como son los climáticos, permite realizar predicciones precisas sobre el comportamiento de las variedades, el cual es necesario para implementar lo que se conoce como agricultura de precisión. Sin embargo, la fruticultura no ha dado el salto hacia la fenómica.

En la  Estación Experimental Agropecuaria (EEA) de San Pedro se ha logrado secuenciar el ADN de más de 250 variedades de duraznero (Aballay et al., 2021, Scientific Reports) disponiendo de una base de datos genómica de 75 gigabases (Gb) de ADN. Esta base permite identificar genes que controlan características del duraznero mediante algoritmos de inteligencia artificial (IA). Además, se dispone de datos climáticos diarios que se toman de forma automática que incluyen: las temperaturas medias, precipitaciones, horas de frío, radiación, etc. Esta información se combina con los datos genómicos y posteriormente, con modelos de IA se predice el comportamiento de las variedades en escenarios climáticos futuros.

Por otro lado, las heladas primaverales son actualmente el mayor problema de los frutales a nivel mundial. Este fenómeno ocurre cuando las flores abiertas (estado “F”) se someten a temperaturas cercanas a los -2.5 °C. Por este motivo, es necesario conocer el número de flores que se encuentran en estado vulnerable ante un pronóstico de heladas, así como también la densidad de flores. Es del interés del INTA determinar el estado fenológico a campo y mejorar para la tolerancia a heladas. Para lograrlo, se empleará IA y visión por computadora.

La visión por computadora es un subdominio de la IA que permite a las máquinas imitar el sistema visual del ser humano. De esta forma, es posible extraer información a partir de imágenes. 

En la actualidad existen algoritmos capaces de detectar y clasificar de forma efectiva plantas a través de imágenes. Estos algoritmos se han utilizado para una gran variedad de aplicaciones, como es el caso de reconocimiento de enfermedades en plantas. Sin embargo, aún no se ha desarrollado un sistema para la extracción de estados fenológicos de la flor de durazneros.

La presente propuesta permitirá automatizar la toma de características de la flor de durazneros, a partir de fotos. De esta forma, se logrará aumentar el caudal de datos y mejorar los modelos de IA existentes.

\section{1.2 Marco de la propuesta}
\label{sec:descripcion}

La ejecución de este proyecto va alineada con el interés del INTA de determinar el estado de la flor de los durazneros a partir de imágenes.

Las especificaciones de las fotos de varetas que se disponen se detallan en la tabla 1. Por otro lado, las características a determinar se indican en la tabla 2. Se deberá evaluar diferentes modelos de IA a fin de definir el óptimo para esta tarea. Se deberá considerar la existencia de bases de datos de imágenes y/o modelos preentrenados disponibles.

\renewcommand{\tablename}{Tabla}
\begin{table}[ht]
\begin{center}
\begin{tabularx}{\textwidth}{| c | c | c | X | }
\hline
\rowcolor[HTML]{C0C0C0}
\multicolumn{4}{ |c| }{Características de las fotos de duraznos} \\ \hline
Formato & Número & Tamaño & Observaciones \\ \hline
JPG     & 250    & 1 MB   & Fotos de varetas de duraznero con flores en           diferentes estados fenológicos. La mayoría está en estado “F” pero también están en estado “E” y “G”. Cada foto tiene una regla. Cada foto tiene entre 10 a 12 varetas. \\ \hline
\end{tabularx}
\caption{}
\label{tab:coches}
\end{center}
\end{table}

\renewcommand{\tablename}{Tabla}
\begin{table}[ht]
\begin{center}
\begin{tabularx}{\textwidth}{| c | X | }
\hline
\rowcolor[HTML]{C0C0C0}
\multicolumn{2}{ |c| }{Características de interés a ser determinadas} \\ \hline

  N° de Flores totales  & Presencia de Flor. \\ \hline
  Tipo de flor          & En las fotos existen dos tipos de flores: campanulácea y rosásea. \\ \hline
  Estado fenológico     & Discriminar entre estado “F” o flor abierta respecto a los demás estados (“E” flor cerrada o “G” flor sin pétalos). \\ \hline
  Densidad de flores    & Determinar la cantidad de flores por cm de vareta. \\ \hline
  
\end{tabularx}
\caption{}
\end{center}
\end{table}

La solución que se propone hará uso de un modelo de detección de objetos óptimo para el problema planteado. Además, para la clasificación de flores, se utilizará un modelo preentrenado que se ajustará a las fotos proporcionadas. Como resultado final, se cumplirán todos los requisitos establecidos en la tabla 2. La figura 1 muestra esta solución a alto nivel.

\begin{figure}[htpb]
\centering 
\includegraphics[width=1\textwidth]{./Figuras/Tesis.drawio.png}
\caption{Diagrama en bloques del sistema.}
\label{fig:diagBloques}
\end{figure}

\pagebreak

\section{2. Identificación y análisis de los interesados}
\label{sec:interesados}

\begin{table}[ht]
%\caption{Identificación de los interesados}
%\label{tab:interesados}
\begin{tabularx}{\linewidth}{@{}|l|X|X|l|@{}}
\hline
\rowcolor[HTML]{C0C0C0} 
Rol           & Nombre y Apellido & Organización 	& Puesto 	\\ \hline
Auspiciante   & \clientename      &\empclientename  &    -    	\\ \hline
Cliente       & \clientename      &\empclientename	&    -   	\\ \hline
Impulsor      & \clientename      &\empclientename  &    -   	\\ \hline
Responsable   & \authorname       & FIUBA        	& Alumno 	\\ \hline
Colaboradores & \cosupname        & \empclientename &    -  	\\ \hline
Orientador    & \supname	      & \pertesupname 	& Director Trabajo final \\ \hline
Equipo        & \cosupname \newline & \empclientename &    -     \\ \hline
Usuario final & INTA              &     INTA        &      -  	\\ \hline
\end{tabularx}
\end{table}



\section{3. Propósito del proyecto}
\label{sec:proposito}

El propósito de este proyecto es desarrollar un algoritmo que identifique flores de los durazneros y sus estados a partir de fotos de varetas, para aumentar el caudal de datos y mejorar los modelos de IA existentes.

\section{4. Alcance del proyecto}
\label{sec:alcance}

El proyecto incluye:
\begin{itemize}
	\item El preprocesamiento de las fotos para entrenar el modelo.
	\item La selección del modelo a entrenar.
	\item La elaboración del notebook de pruebas en Python.
	\item La implementación local del modelo.
\end{itemize}

El proyecto no incluye:
\begin{itemize}
	\item La recolección de datos/fotos.
	\item La integración con otros modelos que utilice el cliente.
\end{itemize}

\section{5. Supuestos del proyecto}
\label{sec:supuestos}
Para el desarrollo del presente proyecto se supone que:
\begin{itemize}
	\item Se contará con suficientes fotos de veretas para entrenar y evaluar el modelo. 
	\item Se contará con soporte para temas no referentes al área de inteligencia artificial que se necesiten para la elaboración de este proyecto.
	\item Se tendrán imágenes con la calidad adecuada para resolver el problema planteado.
	\item Se contará con los recursos de hardware necesarios para el entrenamiento del modelo.
	\item El codigo se desarrollará en una notebook de Google Colab.
\end{itemize}

\section{6. Requerimientos}
\label{sec:requerimientos}

\begin{consigna}{red}
Los requerimientos deben numerarse y de ser posible estar agruparlos por afinidad, por ejemplo:

\begin{enumerate}
	\item Requerimientos funcionales
		\begin{enumerate}
			\item El sistema debe...
			\item Tal componente debe...
			\item El usuario debe poder...
		\end{enumerate}
	\item Requerimientos de documentación
		\begin{enumerate}
			\item Requerimiento 1
			\item Requerimiento 2 (prioridad menor)
		\end{enumerate}
	\item Requerimiento de testing...
	\item Requerimientos de la interfaz...
	\item Requerimientos interoperabilidad...
	\item etc...
\end{enumerate}

Leyendo los requerimientos se debe poder interpretar cómo será el proyecto y su funcionalidad.

Indicar claramente cuál es la prioridad entre los distintos requerimientos y si hay requerimientos opcionales. 

No olvidarse de que los requerimientos incluyen a las regulaciones y normas vigentes!!!

Y al escribirlos seguir las siguientes reglas:
\begin{itemize}
	\item Ser breve y conciso (nadie lee cosas largas). 
	\item Ser específico: no dejar lugar a confusiones.
	\item Expresar los requerimientos en términos que sean cuantificables y medibles.
\end{itemize}

\end{consigna}

\section{7. Historias de usuarios (\textit{Product backlog})}
\label{sec:backlog}

\begin{consigna}{red}
Descripción: En esta sección se deben incluir las historias de usuarios y su ponderación (\textit{history points}). Recordar que las historias de usuarios son descripciones cortas y simples de una característica contada desde la perspectiva de la persona que desea la nueva capacidad, generalmente un usuario o cliente del sistema. La ponderación es un número entero que representa el tamaño de la historia comparada con otras historias de similar tipo.

El formato propuesto es: "como [rol] quiero [tal cosa] para [tal otra cosa]."

Se debe indicar explícitamente el criterio para calcular los \textit{story points} de cada historia
\end{consigna}

\section{8. Entregables principales del proyecto}
\label{sec:entregables}

\begin{consigna}{red}

Los entregables del proyecto son (ejemplo):

\begin{itemize}
	\item Manual de uso
	\item Diagrama de circuitos esquemáticos
	\item Código fuente del firmware
	\item Diagrama de instalación
	\item Informe final
	\item etc...
\end{itemize}

\end{consigna}

\section{9. Desglose del trabajo en tareas}
\label{sec:wbs}

\begin{consigna}{red}
El WBS debe tener relación directa o indirecta con los requerimientos.  Son todas las actividades que se harán en el proyecto para dar cumplimiento a los requerimientos. Se recomienda mostrar el WBS mediante una lista indexada:

\begin{enumerate}
\item Grupo de tareas 1
	\begin{enumerate}
	\item Tarea 1 (tantas h)
	\item Tarea 2 (tantas hs)
	\item Tarea 3 (tantas h)
	\end{enumerate}
\item Grupo de tareas 2
	\begin{enumerate}
	\item Tarea 1 (tantas h)
	\item Tarea 2 (tantas h)
	\item Tarea 3 (tantas h)
	\end{enumerate}
\item Grupo de tareas 3
	\begin{enumerate}
	\item Tarea 1 (tantas h)
	\item Tarea 2 (tantas h)
	\item Tarea 3 (tantas h)
	\item Tarea 4 (tantas h)
	\item Tarea 5 (tantas h)
	\end{enumerate}
\end{enumerate}

Cantidad total de horas: (tantas h)

Se recomienda que no haya ninguna tarea que lleve más de 40 h. 

\end{consigna}

\section{10. Diagrama de Activity On Node}
\label{sec:AoN}

\begin{consigna}{red}
Armar el AoN a partir del WBS definido en la etapa anterior. 

%La figura \ref{fig:AoN} fue elaborada con el paquete latex tikz y pueden consultar la siguiente referencia \textit{online}:

%\url{https://www.overleaf.com/learn/latex/LaTeX_Graphics_using_TikZ:_A_Tutorial_for_Beginners_(Part_3)\%E2\%80\%94Creating_Flowcharts}

\end{consigna}

\begin{figure}[htpb]
\centering 
\includegraphics[width=.8\textwidth]{./Figuras/AoN.png}
\caption{Diagrama de \textit{Activity on Node}.}
\label{fig:AoN}
\end{figure}

Indicar claramente en qué unidades están expresados los tiempos.
De ser necesario indicar los caminos semicríticos y analizar sus tiempos mediante un cuadro.
Es recomendable usar colores y un cuadro indicativo describiendo qué representa cada color, como se muestra en el siguiente ejemplo:



\section{11. Diagrama de Gantt}
\label{sec:gantt}

\begin{consigna}{red}

Existen muchos programas y recursos \textit{online} para hacer diagramas de Gantt, entre los cuales destacamos:

\begin{itemize}
\item Planner
\item GanttProject
\item Trello + \textit{plugins}. En el siguiente link hay un tutorial oficial: \\ \url{https://blog.trello.com/es/diagrama-de-gantt-de-un-proyecto}
\item Creately, herramienta online colaborativa. \\\url{https://creately.com/diagram/example/ieb3p3ml/LaTeX}
\item Se puede hacer en latex con el paquete \textit{pgfgantt}\\ \url{http://ctan.dcc.uchile.cl/graphics/pgf/contrib/pgfgantt/pgfgantt.pdf}
\end{itemize}

Pegar acá una captura de pantalla del diagrama de Gantt, cuidando que la letra sea suficientemente grande como para ser legible. 
Si el diagrama queda demasiado ancho, se puede pegar primero la ``tabla'' del Gantt y luego pegar la parte del diagrama de barras del diagrama de Gantt.

Configurar el software para que en la parte de la tabla muestre los códigos del EDT (WBS).\\
Configurar el software para que al lado de cada barra muestre el nombre de cada tarea.\\
Revisar que la fecha de finalización coincida con lo indicado en el Acta Constitutiva.

En la figura \ref{fig:gantt}, se muestra un ejemplo de diagrama de Gantt realizado con el paquete de \textit{pgfgantt}. En la plantilla pueden ver el código que lo genera y usarlo de base para construir el propio.

\begin{figure}[htbp]
\begin{center}
\begin{ganttchart}{1}{12}
  \gantttitle{2020}{12} \\
  \gantttitlelist{1,...,12}{1} \\
  \ganttgroup{Group 1}{1}{7} \\
  \ganttbar{Task 1}{1}{2} \\
  \ganttlinkedbar{Task 2}{3}{7} \ganttnewline
  \ganttmilestone{Milestone o hito}{7} \ganttnewline
  \ganttbar{Final Task}{8}{12}
  \ganttlink{elem2}{elem3}
  \ganttlink{elem3}{elem4}
\end{ganttchart}
\end{center}
\caption{Diagrama de Gantt de ejemplo}
\label{fig:gantt}
\end{figure}


\begin{landscape}
\begin{figure}[htpb]
\centering 
\includegraphics[height=.85\textheight]{./Figuras/Gantt-2.png}
\caption{Ejemplo de diagrama de Gantt rotado}
\label{fig:diagGantt}
\end{figure}

\end{landscape}

\end{consigna}


\section{12. Presupuesto detallado del proyecto}
\label{sec:presupuesto}

\begin{consigna}{red}
Si el proyecto es complejo entonces separarlo en partes:
\begin{itemize}
	\item Un total global, indicando el subtotal acumulado por cada una de las áreas.
	\item El desglose detallado del subtotal de cada una de las áreas.
\end{itemize}

IMPORTANTE: No olvidarse de considerar los COSTOS INDIRECTOS.

\end{consigna}

\begin{table}[htpb]
\centering
\begin{tabularx}{\linewidth}{@{}|X|c|r|r|@{}}
\hline
\rowcolor[HTML]{C0C0C0} 
\multicolumn{4}{|c|}{\cellcolor[HTML]{C0C0C0}COSTOS DIRECTOS} \\ \hline
\rowcolor[HTML]{C0C0C0} 
Descripción &
  \multicolumn{1}{c|}{\cellcolor[HTML]{C0C0C0}Cantidad} &
  \multicolumn{1}{c|}{\cellcolor[HTML]{C0C0C0}Valor unitario} &
  \multicolumn{1}{c|}{\cellcolor[HTML]{C0C0C0}Valor total} \\ \hline
 &
  \multicolumn{1}{c|}{} &
  \multicolumn{1}{c|}{} &
  \multicolumn{1}{c|}{} \\ \hline
 &
  \multicolumn{1}{c|}{} &
  \multicolumn{1}{c|}{} &
  \multicolumn{1}{c|}{} \\ \hline
\multicolumn{1}{|l|}{} &
   &
   &
   \\ \hline
\multicolumn{1}{|l|}{} &
   &
   &
   \\ \hline
\multicolumn{3}{|c|}{SUBTOTAL} &
  \multicolumn{1}{c|}{} \\ \hline
\rowcolor[HTML]{C0C0C0} 
\multicolumn{4}{|c|}{\cellcolor[HTML]{C0C0C0}COSTOS INDIRECTOS} \\ \hline
\rowcolor[HTML]{C0C0C0} 
Descripción &
  \multicolumn{1}{c|}{\cellcolor[HTML]{C0C0C0}Cantidad} &
  \multicolumn{1}{c|}{\cellcolor[HTML]{C0C0C0}Valor unitario} &
  \multicolumn{1}{c|}{\cellcolor[HTML]{C0C0C0}Valor total} \\ \hline
\multicolumn{1}{|l|}{} &
   &
   &
   \\ \hline
\multicolumn{1}{|l|}{} &
   &
   &
   \\ \hline
\multicolumn{1}{|l|}{} &
   &
   &
   \\ \hline
\multicolumn{3}{|c|}{SUBTOTAL} &
  \multicolumn{1}{c|}{} \\ \hline
\rowcolor[HTML]{C0C0C0}
\multicolumn{3}{|c|}{TOTAL} &
   \\ \hline
\end{tabularx}%
\end{table}


\section{13. Gestión de riesgos}
\label{sec:riesgos}

\begin{consigna}{red}
a) Identificación de los riesgos (al menos cinco) y estimación de sus consecuencias:
 
Riesgo 1: detallar el riesgo (riesgo es algo que si ocurre altera los planes previstos de forma negativa)
\begin{itemize}
	\item Severidad (S): mientras más severo, más alto es el número (usar números del 1 al 10).\\
	Justificar el motivo por el cual se asigna determinado número de severidad (S).
	\item Probabilidad de ocurrencia (O): mientras más probable, más alto es el número (usar del 1 al 10).\\
	Justificar el motivo por el cual se asigna determinado número de (O). 
\end{itemize}   

Riesgo 2:
\begin{itemize}
	\item Severidad (S): 
	\item Ocurrencia (O):
\end{itemize}

Riesgo 3:
\begin{itemize}
	\item Severidad (S): 
	\item Ocurrencia (O):
\end{itemize}


b) Tabla de gestión de riesgos:      (El RPN se calcula como RPN=SxO)

\begin{table}[htpb]
\centering
\begin{tabularx}{\linewidth}{@{}|X|c|c|c|c|c|c|@{}}
\hline
\rowcolor[HTML]{C0C0C0} 
Riesgo & S & O & RPN & S* & O* & RPN* \\ \hline
       &   &   &     &    &    &      \\ \hline
       &   &   &     &    &    &      \\ \hline
       &   &   &     &    &    &      \\ \hline
       &   &   &     &    &    &      \\ \hline
       &   &   &     &    &    &      \\ \hline
\end{tabularx}%
\end{table}

Criterio adoptado: 
Se tomarán medidas de mitigación en los riesgos cuyos números de RPN sean mayores a...

Nota: los valores marcados con (*) en la tabla corresponden luego de haber aplicado la mitigación.

c) Plan de mitigación de los riesgos que originalmente excedían el RPN máximo establecido:
 
Riesgo 1: plan de mitigación (si por el RPN fuera necesario elaborar un plan de mitigación).
  Nueva asignación de S y O, con su respectiva justificación:
  - Severidad (S): mientras más severo, más alto es el número (usar números del 1 al 10).
          Justificar el motivo por el cual se asigna determinado número de severidad (S).
  - Probabilidad de ocurrencia (O): mientras más probable, más alto es el número (usar del 1 al 10).
          Justificar el motivo por el cual se asigna determinado número de (O).

Riesgo 2: plan de mitigación (si por el RPN fuera necesario elaborar un plan de mitigación).
 
Riesgo 3: plan de mitigación (si por el RPN fuera necesario elaborar un plan de mitigación).

\end{consigna}


\section{14. Gestión de la calidad}
\label{sec:calidad}

\begin{consigna}{red}
Elija al menos diez requerientos que a su criterio sean los más importantes/críticos/que aportan más valor y para cada uno de ellos indique las acciones de verificación y validación que permitan asegurar su cumplimiento.

\begin{itemize} 
\item Req \#1: copiar acá el requerimiento.

\begin{itemize}
	\item Verificación para confirmar si se cumplió con lo requerido antes de mostrar el sistema al cliente. Detallar 
	\item Validación con el cliente para confirmar que está de acuerdo en que se cumplió con lo requerido. Detallar  
\end{itemize}

\end{itemize}

Tener en cuenta que en este contexto se pueden mencionar simulaciones, cálculos, revisión de hojas de datos, consulta con expertos, mediciones, etc.  Las acciones de verificación suelen considerar al entregable como ``caja blanca'', es decir se conoce en profundidad su funcionamiento interno.  En cambio, las acciones de validación suelen considerar al entregable como ``caja negra'', es decir, que no se conocen los detalles de su funcionamiento interno.

\end{consigna}

\section{15. Procesos de cierre}    
\label{sec:cierre}

\begin{consigna}{red}
Establecer las pautas de trabajo para realizar una reunión final de evaluación del proyecto, tal que contemple las siguientes actividades:

\begin{itemize}
	\item Pautas de trabajo que se seguirán para analizar si se respetó el Plan de Proyecto original:
	 - Indicar quién se ocupará de hacer esto y cuál será el procedimiento a aplicar. 
	\item Identificación de las técnicas y procedimientos útiles e inútiles que se emplearon, y los problemas que surgieron y cómo se solucionaron:
	 - Indicar quién se ocupará de hacer esto y cuál será el procedimiento para dejar registro.
	\item Indicar quién organizará el acto de agradecimiento a todos los interesados, y en especial al equipo de trabajo y colaboradores:
	  - Indicar esto y quién financiará los gastos correspondientes.
\end{itemize}

\end{consigna}


\end{document}
